\begin{figure}[!t]
\centering
\begin{tikzpicture}
    % ------- style -------
    \tikzset{%
        parenthesized/.style={%
            left delimiter  = (,
            right delimiter = ),
        },
        node distance = 10mu,
    }

    % ------- equation -------
    \matrix[matrix of math nodes, parenthesized] (I) {
        0 & 1 & 1 & 1 & 0 & 0 & 0 \\
        0 & 0 & 1 & 1 & 1 & 0 & 0 \\
        0 & 0 & 0 & 1 & 1 & 1 & 0 \\
        0 & 0 & 0 & 1 & 1 & 0 & 0 \\
        0 & 0 & 1 & 1 & 0 & 0 & 0 \\
        0 & 1 & 1 & 0 & 0 & 0 & 0 \\
        1 & 1 & 0 & 0 & 0 & 0 & 0 \\
    };

    \node (*) [right = of I] {${}*{}$};

    \newcommand\Kmatrix{}
    \foreach \row in {1, ..., 3} {
        \gdef \sep {}
        \foreach \col in {1, ..., 3} {%
            \xdef \Kmatrix {\unexpanded\expandafter{\Kmatrix}\unexpanded\expandafter{\sep}\noexpand \K{\row}{\col}}
            \gdef \sep { \& }
        }
        \xdef \Kmatrix {\unexpanded\expandafter{\Kmatrix}\noexpand\\}
    }
    \matrix[matrix of math nodes, ampersand replacement=\&, left delimiter=[,right delimiter={]}] (K) [right = of *] {
        \Kmatrix
    };

    \node (=) [right = of K] {${}={}$};

    \matrix[matrix of math nodes, parenthesized] (I*K) [right = of {=}] {
        1 & 4 & 3 & 4 & 1 \\
        1 & 2 & 4 & 3 & 3 \\
        1 & 2 & 3 & 4 & 1 \\
        1 & 3 & 3 & 1 & 1 \\
        3 & 3 & 1 & 1 & 0 \\
    };

    % ------- highlighting -------
    \newcommand\rowResult{1}
    \newcommand\colResult{4}

    \begin{scope}[on background layer]
        \newcommand{\padding}{2pt}
        \coordinate (Is-nw) at ([xshift=-\padding, yshift=+\padding] I-\rowResult-\colResult.north west);
        \coordinate (Is-se) at ([xshift=+\padding, yshift=-\padding] I-\the\numexpr\rowResult+\numRowsK-1\relax-\the\numexpr\colResult+\numColsK-1\relax.south east);
        \coordinate (Is-sw) at (Is-nw |- Is-se);
        \coordinate (Is-ne) at (Is-se |- Is-nw);

        \filldraw[red,   fill opacity=.1] (Is-nw) rectangle (Is-se);
        \filldraw[green, fill opacity=.1] (I*K-\rowResult-\colResult.north west) rectangle (I*K-\rowResult-\colResult.south east);

        \draw[blue, dotted] 
            (Is-nw) -- (K.north west)
            (Is-se) -- (K.south east)
            (Is-sw) -- (K.south west)
            (Is-ne) -- (K.north east)
        ;
        \draw[green, dotted] 
            (I*K-\rowResult-\colResult.north west) -- (K.north west)
            (I*K-\rowResult-\colResult.south east) -- (K.south east)
            (I*K-\rowResult-\colResult.south west) -- (K.south west)
            (I*K-\rowResult-\colResult.north east) -- (K.north east)
        ;

        \draw[blue,  fill=blue!10!white] (K.north west) rectangle (K.south east);

        \foreach \row [evaluate=\row as \rowI using int(\row+\rowResult-1)] in {1, ..., \numRowsK} {%
            \foreach \col [evaluate=\col as \colI using int(\col+\colResult-1)] in {1, ..., \numColsK} {%
                    \node[text=blue] at (I-\rowI-\colI.south east) [xshift=-.3em] {\tiny$\times \K{\row}{\col}$};
                }
        }
    \end{scope}

    % ------- labels -------
    \tikzset{node distance=0em}
    \node[below=of I] (I-label) {$\mathbf x_c$};
    \node at (K |- I-label)     {$\mathbf v_c$};
    \node at (I*K |- I-label)   {$\mathbf u_c = \mathbf v_c* \mathbf x$};
\end{tikzpicture}%
\caption[Convolution]{Visualization of a single computation in a $2D$ convolution operation. Local values of the input feature map $\mathbf x_c$ are processed by a $3 \times 3$ kernel and produce a single output value. The process is simultaneously performed in all locations to produce the output feature map $\mathbf u_c$.}\label{fig:convolution}
\end{figure}